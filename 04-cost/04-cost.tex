% (The MIT License)
%
% SPDX-FileCopyrightText: Copyright (c) 2023-2025 Yegor Bugayenko
% SPDX-License-Identifier: MIT

\documentclass{article}
\usepackage{../pmba}
\newcommand*\thetitle{Cost Management}
\begin{document}

\lnTitlePage{4}{10}{naEBGuUfqIk}

\pmbaQuestion
  {In a one-year project, how often do you ask your customer to sign-off the budget?}
  {After every weekend}
  {After every pull request}
  {After every change request}
  {After every phone call}
  {ccb, baseline}

\pmbaQuestion
  {What is the most honest way to bill a customer?}
  {\$100 per hour}
  {\$100 per line of code}
  {\$100 per feature}
  {\$100 per pull request}
  {salary}

\pmbaQuestion
  {How to calculate Cost Performance Indicator (CPI)?}
  {\(\texttt{CPI} = \texttt{EV} / \texttt{AC}\)} % CPI
  {\(\texttt{CPI} = \texttt{EV} / \texttt{PV}\)} % SPI
  {\(\texttt{CPI} = \texttt{EV} - \texttt{PV}\)} % schedule variance
  {\(\texttt{CPI} = \texttt{EV} - \texttt{AC}\)} % cost variance
  {evm}

\pmbaQuestion
  {Which budget looks best of all, in terms of \emph{accuracy} and \emph{precision}?}
  {\$9995.00 +- 2\%}
  {\$9002.99 +- 10\%}
  {\$9900.00 +- 20\%}
  {\$9000.00 +- 100\%}
  {evm}

\pmbaQuestion
  {You are a project manager, a programmer asks you to pay \$500 for a training course about Machine Learning. What do you answer?}
  {``Definitely, not!''}
  {``I have to ask our CFO and the customer''}
  {``Only if you work on weekends''}
  {``How ML is related to our project?!''}
  {school}

\pmbaQuestion
  {You just interviewed two programmers and now have to explain to your boss that you want to hire the one that is more expensive; what is the best argument to use?}
  {She is more enthusiastic about our project}
  {She has longer experience}
  {She has bigger StackOverflow reputation}
  {She worked for our competitor before}
  {value}

\pmbaQuestion
  {Would you allow your team members know the salaries of each other?}
  {Definitely, No!}
  {Only if they are all the same}
  {Only if they are all higher than the market can pay}
  {Definitely, Yes!}
  {salary}

\pmbaQuestion
  {Who is making more money?}
  {OCA}
  {PMP}
  {PhD}
  {CEO}
  {salary}

\plush{
\pptBanner{Earned Value Method}\par
\pptPic{.8}{evm.png}\par
{\scriptsize The picture is taken from PMBOK5, Figure 7-12, page 219.}}

\plush{
\pptBanner{Cost and Staffing Levels}\par
\pptPic{.8}{cost-levels.png}\par
{\scriptsize The picture is taken from PMBOK5, Figure 2-8, page 38.}}

\plush{
\pptBanner{Cost of Changes}\par
\pptPic{.8}{cost-of-changes.png}\par
{\scriptsize The picture is taken from PMBOK5, Figure 2-9, page 39.}}

\plush{
  \pptBanner{Homework:}
  A ``project estimate'' refers to a systematic assessment of the anticipated time, resources, and costs required to complete a specific project. It is derived from a detailed understanding of the project's scope, tasks, and potential challenges, and serves as a foundational tool for budgeting, scheduling, and resource allocation. Accurate project estimates are crucial for setting realistic expectations and ensuring the project's successful completion within its defined constraints.
  --- ChatGPT 4.0
}

\plush{
  \pptBanner{Read this:}
  \nospell{Steve McConnell}, \emph{Software Estimation: Demystifying the Black Art} (2006)\par
  \href{https://www.yegor256.com/2015/06/02/how-to-estimate-software-cost.html}{How Much For This Software?} (2015)\par
  \href{https://www.yegor256.com/2018/01/09/micro-budgeting.html}{Five Stages of Microbudgeting} (2018)\par
  \href{https://www.yegor256.com/2014/10/29/how-much-do-you-cost.html}{How Much Do You Cost?} (2014)\par
  \href{https://www.yegor256.com/2014/04/11/cost-of-loc.html}{How Much Do You Pay Per Line of Code?} (2014)\par
  \href{https://www.yegor256.com/2014/10/21/incremental-billing.html}{Incremental Billing} (2014)\par
}

\end{document}
